\documentclass[11pt,a4paper,onecolumn]{article}
\usepackage[english]{babel}
\usepackage{amsmath}
\numberwithin{equation}{section} %For numbering equations in format chapter.number
\usepackage{amssymb}%math symbols
\usepackage[mathscr]{euscript}
%\usepackage[onehalfspacing]{setspace}
\usepackage{bm}%fette Symbole(Vektoren)
\usepackage{appendix}
\usepackage{cleveref}
\usepackage{graphicx}
\usepackage{subfig}
\usepackage{import} %for importing from other folder
\usepackage{placeins} %Places Figures (FloatBarrier)
\usepackage{mathtools}%dcases
\usepackage{todonotes}
\usepackage{cite}
\usepackage{caption}
\usepackage[margin=1in]{geometry} %For setting the paper boundaries
\usepackage[export]{adjustbox}%Here used for aligning subfloats at the top
\newcommand*{\HomePath}{}
\usepackage{dsfont}

\subimport{\HomePath}{commands.tex}

\graphicspath {{../MyCode/Plots/}}

%\graphicspath {{\HomePath Plots/}}

\captionsetup{labelfont=bf}
\author{Philipp Breul}
\title{Numerics Assignment}
\begin{document}
\maketitle
In this report, we will take a look at different numerical schemes for solving the linear advection problem. We will analyse their properties and compare numerical solutions with analytic. \\ \\
The first chapter will introduce the one-dimensional linear advection equation and very briefly discuss its analytical solution. In the second chapter we will introduce relevant properties of numerical schemes and give an overview which properties various schemes possess. The third chapter will introduce the numerical experiments we run and show the results.  In the last section we discuss these results. \\ \\
Note: Additionally to the schemes presented here, FTCS, CTCS and Semi-Lagrangien were coded up but not shown.
\section{Linear Advection}\label{ch:linear_advection}
In one dimension the linear advection equation takes the form 
\begin{align}
\partial_t\rho(x,t) + \partial_x(u \rho(x,t)) = 0.\label{eq:linear_advection}
\end{align}
With $\rho$ a quantity being advected by the velocity field $u(x,t)$, which in the following we will assume is $u(x,t)\rightarrow u$ a constant.\\
Using the method of charachteristics, we can find an analytical solution for this equation, given by
\begin{align}
\rho(x,t) = \rho_0(x-ut),\quad \rho_0(x) = \rho(x,0).
\end{align}
Although having an analytical solution, we want to solve \cref{eq:linear_advection} numerically to compare the results with the known solution and thus judge their accuracy. \\
For a numerical simulation we need to discretise our field $\rho(x,t)$ both in space and time, $\rho_j^n$, where two spatial points $\rho_j^n$ and $\rho_{j+1}^n$ are separated by $\Delta x$ and two points in time by $\Delta t$ respectively. A numerical scheme is then concerned with obtaining the solution $\rho_j^{n+1}$ at the next time point. For finite-difference methods  methods, this is done by approximating the derivatives in the pde with finite differences and then solving the equation for $\rho_j^{n+1}$ . Other classes of numerical schemes (e.g. the semi-Lagrangien) operate differently.\\
The report and associated code assumes periodic boundary conditions, which means, that for $\rho_j^{n}$ with $j \in [0,1,... N] $ we have $\rho_{N+1}^n=\rho_0^n$ and $\rho_{-1}^n=\rho_N^n$.
\section{Numerical Schemes}
In this chapter we will first take a look at three different schemes, the simple \textit{FTCS}-scheme, the implicit \textit{BTCS}-scheme and the more involved \textit{semi-Lagrangien} scheme. We will then introduce some desired properties of our numerical schemes and finally give an overview which scheme possesses which property.
\subsection{FTCS}
%The Forward in Time Centered in Space (FTCS) is a finite-difference scheme, so as explained in \cref{ch:linear_advection}, we are concerned with discretising the derivatives with finite differences. This scheme approximates the temporal and spatial derivatives as
%\begin{align}
%\pder{\rho^n_j}{t}&=\frac{\rho_j^{n+1}-\rho_j^n}{\Delta t},\\
%\pder{\rho^n_j}{x}&=\frac{\rho_{j+1}^{n}-\rho_{j-1}^n}{2\Delta x}.
%\end{align}
%When plugging this into \cref{eq:linear_advection} and solving the resulting equation for $\rho^{n+1}_j$, we get
%\begin{align}					
%\rho^{n+1}_j = \rho_j^{n}-\frac c2\left(\rho^n_{j+1}-\rho_{j-1}^n\right),
%\end{align}
%with $c=\frac{u\Delta t}{\Delta x}$, the Courant number. It can be viewed as a non-dimensionalised velocity. So with the field $\rho$ at time step $n$, we can calculate $\rho$ at time step $n+1$.
\subsection{CTCS}
\subsection{BTCS}
%The Backward in Time Centered in Space  (BTCS) is a finite-difference scheme as well. The derivatives approximate to
%\begin{align}
%\pder{\rho^{n+1}_j}{t}&=\frac{\rho_j^{n+1}-\rho_j^{n}}{\Delta t},\\
%\pder{\rho^{n+1}_j}{x}&=\frac{\rho_{j+1}^{n+1}-\rho_{j-1}^{n+1}}{2\Delta x}.
%\end{align}
%Plugging this into equation \cref{eq:linear_advection} and solving for $\rho^{n+1}_j$, leaves us with
%\begin{align}
%\rho_j^{n+1}=\rho_j^n-\frac c2\left(\rho_{j+1}^{n+1}-\rho_{j-1}^{n+1}\right).\label{eq:BTCS}
%\end{align}
%This obviously poses a problem, since the solution at the next time step should not depend on itself. But we can solve  \cref{eq:BTCS} for $\rho_j^n$.
%This can then be written down as a matrix equation
%\begin{align}
%\bm M \begin{pmatrix}
%\rho^{n+1}_0\\
%\rho^{n+1}_1\\
%\vdots\\
%\rho^{n+1}_N\\
%\end{pmatrix}
%=
%\begin{pmatrix}
%\rho^{n}_0\\
%\rho^{n}_1\\
%\vdots\\
%\rho^{n}_N\\
%\end{pmatrix}
%\end{align}
%We can solve for the solution at $n+1$ by either multiplying by $M^{-1}$, or solving the system of linear equations (numerically advantageous).
\subsection{LaxWendroff}
\subsection{Properties of numerical schemes}
We will now describe different properties of numerical schemes.
\paragraph{Stability}
Stability means that the error stays finite no matter for arbitrary number of time steps. Schemes can be unconditionally stable, meaning they are stable for any time step, no matter how large. The contrary is unconditionally unstable, no matter the time step, the error will always tend to infinity. The third option is conditionally stable, where the scheme is stable or unstable depending on the size of the time step (or in our case the Courant number $c$).\\
It is important to note, that even if a scheme is stable, this does not mean, that the solution is very precise, it just does not tend to infinity for infinite times. 
\paragraph{Moment conservation}
If the PDE conserves mass (and higher moments), it would be preferable if the numerical scheme preserved them too. Here we will concern our selves with 
\begin{alignat}{2}
\text{(mass)}& \qquad M&&=\sum_{i=0}^{N-1} \Delta x\rho_i \\
\text{(variance)}&\qquad V&&=\sum_{i=0}^{N-1} \Delta x\rho_i ^2- M^2
\end{alignat}
the mass and the variance. Since the linear advection is a conservation law, we would prefer our schemes to preserve these quantities, so that they are independet of the time step.
\paragraph{Numerical Diffusion}
Some schemes show a diffusive behaviour even though the analytical solution does not diffuse, thus numerical diffusion. Diffusive schemes reduce the variance over time, which is a quantitative way to determine if a scheme is diffusive. \todo{Find Citation} 
\paragraph{Order of Accuracy}
This describes the dependence of the error on the discretisation. One usually gives the first lead order of this dependence,  error $\propto (\Delta x)^n$ would mean that the scheme is $n$-th order accurate. This is an important property of a scheme since it tells us how our error will behave when changing resolution.
\section{Numerical Experiments}
\subsection{Experimental Design}
We will use two main experiments to analyse the properties of our numerical schemes introduced above. \\
First we we simply simply evolve our initial condition in time and meanwhile monitor mass, variance and $l_2$-error at every time step. 
With this we can test multiple properties. First, we can obviously test, if a scheme is mass and variance preserving. But we also know, that a diffusive scheme reduces the variance\todo{Find Citation}. So together with a visual test of the advection, we can judge if a scheme has numerical diffusion. \\
The error over time gives us information about the stability of the scheme, if it stays bounded, the scheme is stable. \\
We will vary the Courant number in order to see when conditionally stable schemes become unstable. We will also use different initial conditions (smooth and discontinuous).\\ \\
The second experiment is supposed to test accuracy of the schemes. We want to investigate the error depending on the discretisation $\Delta x$. When adjusting the number of grid points, we have to adjust the number of time steps accordingly, so that the quotient of the two is constant. We have to do that so that the  Courant number stays fixed for same velocity field $u$. \\
We take the $l_2$-error at the end of each run and plot it against the discretisation to see how the error scales. 
\subsection{Results}
In \cref{fig:gauss_stable} a gaussian curve was propagated in time for a Courant number of $c=0.2$. In the first picture we see the final state. Wee see, that Lax-Wendroff seems to perform better than BTCS which in turn is better than FTBS. This is also reflected in the second frame where the  $l_2$ error norms are shown and one can quantitatively see that the error of LaxWendroff is lower then BTCS which is lower then FTBS. We will discuss the oscillation of the error norm in \cref{ch:discussion}. \\
From the error curves we can also tell, that they will probably stay bounded and converge, which we would expect, since the schemes are stable in this regime. \\
While LaxWendroff and FTBS seem to be concerving mass in the third frame, BTCS does not. But the effect is minor and could be a numerical artifact from solving the linear equation system. \\
But we find clearly that while Lax-Wendroff is variance preserving, the other two schemes are not. This means, that the other two schemes are diffusive. This is in agreement to the visual diffusion we see in the first figure.\\ \\
In \cref{fig:step_stable} we wanted to test the performance of the schemes under similar conditions but with discontinuous initial condition. For both LaxWendroff and BTCS we can observe numerical dispersion which leads to oscillatory behaviour near the discontinuity, but in this report we will not focus on dispersion. Otherwise the performance is similar to the case of smooth initial conditions. All schemes seem stable, judging by their error curves, although LaxWendroff does not seem to perform better than BTCS anymore.\\
Except from the small deviation from BTCS, the schemes still preserve mass. \\
But surprisingly, while FTBS and BTCS are still loosing variance over time, this effect is not as strong as in \cref{fig:gauss_stable}. A short discussion can be found in \cref{ch:discussion}. \\ \\
We also wanted to show that schemes become unstable. So we used a Courant number of $c=1.4$ in \cref{fig:gauss_unstable}. By eye one can already tell that the solution looks terrible for FTBS and LaxWendroff, and the error curve is clearly divergent for these two schemes. The BTCS scheme on the other hand seems to be stable as expected.\\
Interestingly LaxWendroff does not seem to preserve neither mass nor variance any more. And we tested that for more propagation steps FTBS would not preserve mass either.\\
The brake down off these properties is not surprising, since the two schemes are unstable.  \\ \\
In \cref{fig:accuracy} the second experiment we ran can be seen, where we wanted to test the accuracy of the schemes. These follow nicely the theoretically expected scaling laws but deviate for larger discretisations. This is expected, since the order of accuracy is determined by the first non vanishing Taylor term and assuming all higher order terms are comparably small. But for ever larger $\Delta x$, the higher order terms become non-negligible and thus the curves deviate. 
\FloatBarrier
\begin{figure}
\centering
\includegraphics[width = 0.91\textwidth]{gauss_c0_2dx0_08.pdf}
\caption{Shown is a gaussian initial condition, $\sqrt{\pi}\mathcal{N}(10,1/2)$, on a grid with 250 gridpoints which was propagated with different schemes and Courant number of $c=0.2$ for 560 timesteps. The first figure shows the final configuration, the second the $l_2$-error norm against time steps, the third picture the relative mass difference and the fourth picture the relative variance difference to the initial conditions.}\label{fig:gauss_stable}
\includegraphics[width = 0.91\textwidth]{step_c0_2dx0_08.pdf}
\caption{Shown is an initial step function for $\rho_0 = 1,  \text{if}\ x \in [0,10] $, otherwise same set up as in \cref{fig:gauss_stable} for comparison. }\label{fig:step_stable}
\end{figure}
\FloatBarrier
\begin{figure}
\centering
\includegraphics[width = 0.91\textwidth]{gauss_c1_4dx0_16.pdf}
\caption{Shown is a gaussian initial condition, $\sqrt{\pi}\mathcal{N}(10,1/2)$, on a grid with 125 gridpoints, a Courant number of $c=1.4$ and 40 timesteps, so that the quantities are  consistent and we can compare. One can easily see that the LaxWendroff and FTBS become unstable since their error norms start to diverge, while the error for the BTCS scheme is still higher then in \cref{fig:gauss_stable}, it still stays bounded, since this scheme is stable.}\label{fig:gauss_unstable}
\includegraphics[width = 0.6\textwidth]{OrderAccuracy_c0_2.pdf}
\caption{Shown are $l_2$ error norms for different discretisations $\Delta x$. We want to illustrate the accuracy of a scheme, which is the leading term dependence of the error on the discretisation. For the creation of this plot a gaussian initial condition was used, $\sqrt{\pi}\mathcal{N}(10,1/2)$ and a Courant number of $c=0.2$. The time steps were equally adjusted to the number of grid points. The discretisation $\Delta x$ was normalised by the whole domain $X$. For small $\Delta x$, the curves follow nicely the theoretically expected scaling behaviour.}\label{fig:accuracy}
\end{figure}
\FloatBarrier

\section{Discussion and Conclusion}
Disucuss if schemes show the properties we analytically expect them to have and compare which advantages/disadvantages the schemes have.
\FloatBarrier
\bibliography{mybib}{}
\bibliographystyle{plain}
\end{document}
