\section{Discussion and Conclusion}\label{ch:discussion}
We will briefly recap on what we have found in the last section and link it to the analytically expected properties. We then discuss some questions that were raised and then give a final conclusion. \\ \\
We found the LaxWendroff scheme to behave like theoretically expected. It is second order accurate and stable for $|c|<1$ (here we only showed two examples for $c$, but more were tested). It is also mass variance preserving, which means it is not diffusive. This holds for smooth and discontinuous initial conditions, as long as we are in the stable regime of the scheme. \\ \\
The FTBS scheme behaves as well as expected. It is first order accurate and we found it to be stable for $0\leq c \leq 1 $. It preserves mass but not variance, which fits, since it is a diffusive scheme. Again, this holds while being in the stable regime of the scheme.  \\ \\
The BTCS scheme is first order accurate, as we expected. For the tested values of $c$ it was always stable, which seems to confirm the theoretical prediction of unconditional stability. It is also not variance preserving, which makes it diffusive.\\
 But we were a little bit startled to see that it is not mass preserving, since we expected this from theory. But the effect is very little, on the order of $-14$. Therefore we believe that it has something to do with floating point errors in the solution of the system of linear equations. But that it seams to be a systematic error rather than random, still seems odd. A more precise investigation would be necessary. \\ \\
 For all schemes we observed oscillations in the $l_2$-error norm. This is probably due to the discretisation of the $l_2$ norm, since the period of oscillation seems to be $1/c$. \\ \\
 
 In this report we took a look at different numerical schemes for solving the linear advection equation. We compared their theoretically predicted properties to what is observed numerically and found that the schemes behave in almost all respects as expected. LaxWendroff seems to be advantageous compared to the other two as long as it is stable. It has the highest order of accuracy and does not suffer from numerical diffusion. For Courant numbers bigger then $|1|$, BTCS would be preferable, since it is the only stable scheme. 