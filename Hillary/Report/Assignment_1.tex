\documentclass[11pt,a4paper,onecolumn]{article}
\usepackage[english]{babel}
\usepackage{amsmath}
\numberwithin{equation}{section} %For numbering equations in format chapter.number
\usepackage{amssymb}%math symbols
\usepackage[mathscr]{euscript}
%\usepackage[onehalfspacing]{setspace}
\usepackage{bm}%fette Symbole(Vektoren)
\usepackage{appendix}
\usepackage{cleveref}
\usepackage{graphicx}
\usepackage{subfig}
\usepackage{import} %for importing from other folder
\usepackage{placeins} %Places Figures (FloatBarrier)
\usepackage{mathtools}%dcases
\usepackage{todonotes}
\usepackage{cite}
\usepackage{caption}
\usepackage[margin=1in]{geometry} %For setting the paper boundaries
\usepackage[export]{adjustbox}%Here used for aligning subfloats at the top
\newcommand*{\HomePath}{}
\usepackage{dsfont}

\subimport{\HomePath}{commands.tex}

\graphicspath {{Plots/}}

\graphicspath {{\HomePath Plots/}}

\captionsetup{labelfont=bf}
\author{Philipp Breul}
\title{Numerics Assignment}
\begin{document}
\maketitle
In this report, we will take a look at different numerical schemes for solving the linear advection problem. We will analyse their properties and compare this to numerical simulation. \\ \\
The first chapter will introduce the one-dimensional linear advection equation and very briefly discuss its analytical solution. In the second chapter we will introduce relevant properties of numerical schemes and give an overview properties various schemes possess. The third chapter will then compare this analysis to numerical results. 
\section{Linear Advection}\label{ch:linear_advection}
In one dimension the linear advection equation takes the form 
\begin{align}
\partial_t\rho(x,t) + \partial_x(u \rho(x,t)) = 0.\label{eq:linear_advection}
\end{align}
With $\rho$ a quantity being advected by the velocity field $u(x,t)$, which in the following we will assume is $u(x,t)\rightarrow u$ a constant.\\
Using the method of charachteristics, we can find an analytical solution for this equation, given by
\begin{align}
\rho(x,t) = \rho_0(x-ut),\quad \rho_0(x) = \rho(x,0).\label{eq:analytical_sol}
\end{align}
Although having an analytical solution, we want to solve \cref{eq:linear_advection} numerically to compare the results with the known solution and thus judge their accuracy. \\
For a numerical simulation we need to discretise our field $\rho(x,t)$ both in space and time, $\rho_j^n$, where two spatial points $\rho_j^n$ and $\rho_{j+1}^n$ are separated by $\Delta x$ and two points in time by $\Delta t$ respectively. A numerical scheme is then concerned with obtaining the solution $\rho_j^{n+1}$ at the next time point. For finite-difference methods  methods, this is done by approximating the derivatives in the pde with finite differences and then solving the equation for $\rho_j^{n+1}$ . Other classes of numerical schemes (e.g. the semi-Lagrangien) operate differently.\\
The report and associated code assumes periodic boundary conditions, which means, that for $\rho_j^{n}$ with $j \in [0,1,... N] $ we have $\rho_{N+1}^n=\rho_0^n$ and $\rho_{-1}^n=\rho_N^n$.
\section{Numerical Schemes}
In this chapter we will first take a look at three different schemes, the simple \textit{FTCS}-scheme, the implicit \textit{BTCS}-scheme and the more involved \textit{semi-Lagrangien} scheme. We will then introduce some desired properties of our numerical schemes and finally give an overview which scheme possesses which property.
\subsection{FTCS}
The Forward in Time Centered in Space (FTCS) is a finite-difference scheme, so as explained in \cref{ch:linear_advection}, we are concerned with discretising the derivatives with finite differences. This scheme approximates the temporal and spatial derivatives as
\begin{align}
\pder{\rho^n_j}{t}&=\frac{\rho_j^{n+1}-\rho_j^n}{\Delta t},\\
\pder{\rho^n_j}{x}&=\frac{\rho_{j+1}^{n}-\rho_{j-1}^n}{2\Delta x}.
\end{align}
When plugging this into \cref{eq:linear_advection} and solving the resulting equation for $\rho^{n+1}_j$, we get
\begin{align}					
\rho^{n+1}_j = \rho_j^{n}-\frac c2\left(\rho^n_{j+1}-\rho_{j-1}^n\right),
\end{align}
with $c=\frac{u\Delta t}{\Delta x}$, the Courant number. It can be viewed as a non-dimensionalised velocity. So with the field $\rho$ at time step $n$, we can calculate $\rho$ at time step $n+1$.
\subsection{BTCS}
The Backward in Time Centered in Space  (BTCS) is a finite-difference scheme as well. The derivatives approximate to
\begin{align}
\pder{\rho^{n+1}_j}{t}&=\frac{\rho_j^{n+1}-\rho_j^{n}}{\Delta t},\\
\pder{\rho^{n+1}_j}{x}&=\frac{\rho_{j+1}^{n+1}-\rho_{j-1}^{n+1}}{2\Delta x}.
\end{align}
Plugging this into equation \cref{eq:linear_advection} and solving for $\rho^{n+1}_j$, leaves us with
\begin{align}
\rho_j^{n+1}=\rho_j^n-\frac c2\left(\rho_{j+1}^{n+1}-\rho_{j-1}^{n+1}\right).\label{eq:BTCS}
\end{align}
This obviously poses a problem, since the solution at the next time step should not depend on itself. But we can solve  \cref{eq:BTCS} for $\rho_j^n$.
This can then be written down as a matrix equation
\begin{align}
\bm M \begin{pmatrix}
\rho^{n+1}_0\\
\rho^{n+1}_1\\
\vdots\\
\rho^{n+1}_N\\
\end{pmatrix}
=
\begin{pmatrix}
\rho^{n}_0\\
\rho^{n}_1\\
\vdots\\
\rho^{n}_N\\
\end{pmatrix}
\end{align}
We can solve for the solution at $n+1$ by either multiplying by $M^{-1}$, or solving the system of linear equations (numerically advantageous).
\subsection{Semi Lagrangien}
\subsubsection{Properties of numerical schemes}
\subsection{Stability}
\subsubsection{Moment conservation}
\subsubsection{Numerical Diffusion and Dispersion}

\section{Numerical Experiments}
\subsubsection{Experimental Design}
We will use two main experiments to analyse the properties of our numerical schemes introduced above. \\
First we we simply simply evolve our initial condition in time and meanwhile monitor mass, variance and $l_2$-error at every time step. 
With this we can test multiple properties. First, we can obviously test, if a scheme is mass and variance preserving. But we also know, that a diffusive scheme reduces the variance\todo{Find Citation}. So together with a visual test of the advection, we can judge if a scheme has numerical diffusion. \\
The error over time gives us information about the stability of the scheme, if it stays bounded, the scheme is stable. \\
We will vary the Courant number in order to see when conditionally stable schemes become unstable. We will also use different initial conditions (smooth and discontinuous).\\ \\
The second experiment is supposed to test accuracy of the schemes. We want to investigate the error depending on the discretisation $\Delta x$. When adjusting the number of grid points, we have to adjust the number of time steps accordingly, so that the quotient of the two is constant. We have to do that so that the  Courant number stays fixed for same velocity field $u$. \\
We take the $l_2$-error at the end of each run and plot it against the discretisation to see how the error scales.
\subsection{Results}
Here come the results.
\section{Discussion and Conclusion}\label{ch:discussion}
We will briefly recap on what we have found in the last section and link it to the analytically expected properties. We then discuss some questions that were raised and then give a final conclusion. \\ \\
We found the LaxWendroff scheme to behave like theoretically expected. It is second order accurate and stable for $|c|<1$ (here we only showed two examples for $c$, but more were tested). It is also mass and variance preserving, which means it is not diffusive. This holds for smooth and discontinuous initial conditions, as long as we are in the stable regime of the scheme. \\ \\
The FTBS scheme behaves as well as expected. It is first order accurate and we found it to be stable for $0\leq c \leq 1 $. It preserves mass but not variance, which fits, since it is a diffusive scheme. Again, this holds while being in the stable regime of the scheme.  \\ \\
The BTCS scheme is first order accurate, as we expected. For the tested values of $c$ it was always stable, which seems to confirm the theoretical prediction of unconditional stability. It is also not variance preserving, which makes it diffusive.\\
 But we were a little bit startled to see that it is not mass preserving, since we expected this from theory. But the effect is very little, on the order of $10^{-14}$. Therefore we believe that it has something to do with floating point errors in the solution of the system of linear equations. But that it seams to be a systematic error rather than random, still seems odd. A more precise investigation would be necessary. \\ \\
 For all schemes we observed oscillations in the $l_2$-error norm. This is probably due to the discretisation of the $l_2$ norm, since the period of oscillation seems to be $1/c$. \\ \\
 
 In this report we took a look at different numerical schemes for solving the linear advection equation. We compared their theoretically predicted properties to what is observed numerically and found that the schemes behave in almost all respects as expected. In its range of stability LaxWendroff is advantageous compared to the other two. It has the highest order of accuracy and does not suffer from numerical diffusion. For Courant numbers bigger then $|1|$, BTCS would be preferable, since it is the only stable scheme. 
\end{document}
