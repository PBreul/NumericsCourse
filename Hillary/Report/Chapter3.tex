\section{Numerical Schemes}
In this chapter we will first take a look at three different schemes, the simple \textit{FTCS}-scheme, the implicit \textit{BTCS}-scheme and the more involved \textit{semi-Lagrangien} scheme. We will then introduce some desired properties of our numerical schemes and finally give an overview which scheme possesses which property.
\subsection{FTCS}
%The Forward in Time Centered in Space (FTCS) is a finite-difference scheme, so as explained in \cref{ch:linear_advection}, we are concerned with discretising the derivatives with finite differences. This scheme approximates the temporal and spatial derivatives as
%\begin{align}
%\pder{\rho^n_j}{t}&=\frac{\rho_j^{n+1}-\rho_j^n}{\Delta t},\\
%\pder{\rho^n_j}{x}&=\frac{\rho_{j+1}^{n}-\rho_{j-1}^n}{2\Delta x}.
%\end{align}
%When plugging this into \cref{eq:linear_advection} and solving the resulting equation for $\rho^{n+1}_j$, we get
%\begin{align}					
%\rho^{n+1}_j = \rho_j^{n}-\frac c2\left(\rho^n_{j+1}-\rho_{j-1}^n\right),
%\end{align}
%with $c=\frac{u\Delta t}{\Delta x}$, the Courant number. It can be viewed as a non-dimensionalised velocity. So with the field $\rho$ at time step $n$, we can calculate $\rho$ at time step $n+1$.
\subsection{CTCS}
\subsection{BTCS}
%The Backward in Time Centered in Space  (BTCS) is a finite-difference scheme as well. The derivatives approximate to
%\begin{align}
%\pder{\rho^{n+1}_j}{t}&=\frac{\rho_j^{n+1}-\rho_j^{n}}{\Delta t},\\
%\pder{\rho^{n+1}_j}{x}&=\frac{\rho_{j+1}^{n+1}-\rho_{j-1}^{n+1}}{2\Delta x}.
%\end{align}
%Plugging this into equation \cref{eq:linear_advection} and solving for $\rho^{n+1}_j$, leaves us with
%\begin{align}
%\rho_j^{n+1}=\rho_j^n-\frac c2\left(\rho_{j+1}^{n+1}-\rho_{j-1}^{n+1}\right).\label{eq:BTCS}
%\end{align}
%This obviously poses a problem, since the solution at the next time step should not depend on itself. But we can solve  \cref{eq:BTCS} for $\rho_j^n$.
%This can then be written down as a matrix equation
%\begin{align}
%\bm M \begin{pmatrix}
%\rho^{n+1}_0\\
%\rho^{n+1}_1\\
%\vdots\\
%\rho^{n+1}_N\\
%\end{pmatrix}
%=
%\begin{pmatrix}
%\rho^{n}_0\\
%\rho^{n}_1\\
%\vdots\\
%\rho^{n}_N\\
%\end{pmatrix}
%\end{align}
%We can solve for the solution at $n+1$ by either multiplying by $M^{-1}$, or solving the system of linear equations (numerically advantageous).
\subsection{LaxWendroff}
\subsection{Properties of numerical schemes}
We will now describe different properties of numerical schemes.
\paragraph{Stability}
Stability means that the error stays finite no matter for arbitrary number of time steps. Schemes can be unconditionally stable, meaning they are stable for any time step, no matter how large. The contrary is unconditionally unstable, no matter the time step, the error will always tend to infinity. The third option is conditionally stable, where the scheme is stable or unstable depending on the size of the time step (or in our case the Courant number $c$).\\
It is important to note, that even if a scheme is stable, this does not mean, that the solution is very precise, it just does not tend to infinity for infinite times. 
\paragraph{Moment conservation}
If the PDE conserves mass (and higher moments), it would be preferable if the numerical scheme preserved them too. Here we will concern our selves with 
\begin{alignat}{2}
\text{(mass)}& \qquad M&&=\sum_{i=0}^{N-1} \Delta x\rho_i \\
\text{(variance)}&\qquad V&&=\sum_{i=0}^{N-1} \Delta x\rho_i ^2- M^2
\end{alignat}
the mass and the variance. Since the linear advection is a conservation law, we would prefer our schemes to preserve these quantities, so that they are independet of the time step.
\paragraph{Numerical Diffusion}
Some schemes show a diffusive behaviour even though the analytical solution does not diffuse, thus numerical diffusion. Diffusive schemes reduce the variance over time, which is a quantitative way to determine if a scheme is diffusive. \todo{Find Citation} 
\paragraph{Order of Accuracy}
This describes the dependence of the error on the discretisation. One usually gives the first lead order of this dependence,  error $\propto (\Delta x)^n$ would mean that the scheme is $n$-th order accurate. This is an important property of a scheme since it tells us how our error will behave when changing resolution.